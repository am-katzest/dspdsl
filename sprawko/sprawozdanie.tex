\documentclass[12pt]{article}
\usepackage[T1]{fontenc}
\usepackage[T1]{polski}
\newcommand{\BibTeX}{{\sc Bib}\TeX}
\usepackage{graphicx}
\usepackage{amsmath} \usepackage{amssymb} \usepackage{amsfonts}
\usepackage[utf8]{inputenc}
\usepackage{minted}
\newcommand{\cljt}[1]{\mintinline{clojure}{#1}}
\usepackage{tikz-cd}
\setlength{\textheight}{21cm}

\title{{\bf Zadanie nr 1 - Generacja sygnału i szumu}\linebreak
	Cyfrowe Przetwarzanie Sygnałów}
\author{Jakub Mileczarek idk \and Imię Nazwisko, 236551}
\date{2023-03-06} % jak zdążymy xd

\begin{document}
\clearpage\maketitle
\thispagestyle{empty}
\newpage
\setcounter{page}{1}
\section{Cel zadania}

%Opis celu zadania (proszę nie przepisywać treści instrukcji!).\\
%Sprawozdanie należy wykonać na podstawie\\
%szablonu \LaTeX-owego \texttt{sprawozdanie-wzor.tex}.


\section{Wstęp teoretyczny}

%Krótki opis wykorzystywanych metod~\cite{dowolna_etykieta_artykulu}. Proszę nie umieszczać ogólnie znanych z literatury
%wzorów oraz definicji. Należy podać jaka metoda została zastosowana, dlaczego oraz podać wykorzystaną literaturę (korzystając z odwołań do pozycji bibliografii~\cite{dowolna_etykieta_ksiazki}).\\
%Przygotowując bibliografię należy korzystać z podanego\\
%szablonu \BibTeX-owego \texttt{bibliografia-wzor.bib}.

%%%%%%%%%%%%%%%%%%%%%%%%%%%%%%%%%%%%%%%%%%%%%%%%%%%%%%%%%%%%%%%%%%%%%%%%%%%%%%%%%%%%%%%%%%%%%%%%%%%%%%%%%%%%%%%%%
% PODROZDZIA\xC5\x81 PT. ZALACZNIKI
%%%%%%%%%%%%%%%%%%%%%%%%%%%%%%%%%%%%%%%%%%%%%%%%%%%%%%%%%%%%%%%%%%%%%%%%%%%%%%%%%%%%%%%%%%%%%%%%%%%%%%%%%%%%%%%%%
\section{Wstęp}
zamiast programu z graficznym lub tekstowym interfejsem użytkownika zaimplementowaliśmy malutki, prosty w obsłudze język dziedzinowy.

\section{implementacja}
Program został zaimplementowany w \texttt{clojure}, funkcyjnym, lispopodobnym języku hostowanym na \texttt{JVM}.
\subsection{reprezentacje}
program korzysta wewnętrznie z 4 różnych reprezentacji sygnału:
\begin{itemize}
	\item		\cljt{:spec} \\
	      czytelny dla człowieka słownik zawierający specyfikację, większość elementów ma dość rozsądne wartości domyślne więc może być pominięta, n.p:\\
	      \cljt{{:function :triangle :fill 0.4 :amplitude 0.4}} \\
	      \cljt{{:function :sin :phase 0.3 :duration 3}} \\
	      dla wygody niektóre klucze posiadają skrócone wersje: \\
	      \cljt{{:f :sin :A 2 :p 3 :s 2 :e 5}}

	\item		\cljt{:fancy} \\
	      reprezentacja sygnału ciągłego jako słownik zawierający informacje o początku i końcu sygnału oraz funkcję zwracającą wartość sygnału w danym czasie
	\item		\cljt{:discrete} \\
	      zdyskretyzowana forma sygnału, zawiera informacje o częstotliwości próbkowania, offsecie, ilości próbek, i próbki
	\item		\cljt{:file} \\
	      nazwa pliku zawierającego sygnał
\end{itemize}
funkcje do manipulacji wspierają jedynie \cljt{:discrete} i \cljt{:fancy} (nie wszystkie), inne reprezentacje są w miarę potrzeb konwertowane do którejś ze wspieranych form
\[\begin{tikzcd}[%
			%row sep = 20mm, column sep = 20mm, % both optional, adapt as you please
			cells={nodes={circle}},
			every arrow/.append style={shorten <= 1mm, shorten >= 1mm}] % both optional. Default connects the arrow directly to the circles
		:file \arrow[shift left]{r} &
		:discrete \arrow[shift left]{l} &
		:fancy \arrow[shift left]{l} &
		:spec \arrow{l} &
	\end{tikzcd}\]


\section{Instrukcja}
\subsection{wprowadzanie sygnałów}
użytkownik wprowadza sygnały (poza impulsem jednostkowym) w formie \cljt{:spec} poprzez mapę zawierającą
chciane parametry
\begin{itemize}
	\item \cljt{:function}, \cljt{:fun}, \cljt{:fn}, \cljt{:f} -- jedno z słów kluczowych oznaczających funkcje
	      \begin{itemize}
		      \item \cljt{:sin} sinusoida
		      \item \cljt{:sin-half} sinusoida wyprostowana jednopołówkowo
		      \item \cljt{:sin-double} sinusoida wyprostowana dwupołuwkowo
		      \item \cljt{:square} sygnał prostokątny (\cljt{:fill} -- stopień wypełnienia)
		      \item \cljt{:square-sym} sygnał prostokątny symetryczny
		      \item \cljt{:triangle} sygnał trójkątny (\cljt{:fill} -- część okresu przeznaczona na zbocze narastające)
		      \item \cljt{:jump} skok jednostkowy (\cljt{:fill} -- czas skoku)
		      \item \cljt{:noise} szum (-1 -- 1)
		      \item \cljt{:noise-gauss} szum gaussowski
		      \item \cljt{:noise-impulse} szum impulsowy (\cljt{:fill} -- prawdopodobieństwo)

	      \end{itemize}
\end{itemize}

\subsection{prezentacja}
funkcje znajdujące się w przestrzeni nazw \cljt{graph}
\begin{itemize}
	\item \cljt{show} -- wyświetl sygnał jako punkty lub krzywą w zależności od typu sygnału
	\item \cljt{histogram} -- wyświetla histogram sygnału dyskretnego
	\item \cljt{stat} -- zwraca mapę zawierającą parametry sygnału dyskretnego
\end{itemize}

\subsection{zapisywanie i wczytywanie sygnałów do pliku}
do zapisywania sygnału do pliku służy procedura \cljt{write} biorąca dwa argumenty, nazwę pliku i sygnał;
\cljt{(write "sygnał" {:function :sin})}
wczytywanie odbywa się automatycznie;
\cljt{(show (fop + "sygnał" {:function :square}))}.

%%%%%%%%%%%%%%%%%%%%%%%%%%%%%%%%%%%%%%%%%%%%%%%%%%%%%%%%%%%%%%%%%%%%%%%%%%%%%%%%%%%%%%%%%%%%%%%%%%%%%%%%%%%%%%%%%
% BIBLIOGRAFIA
%%%%%%%%%%%%%%%%%%%%%%%%%%%%%%%%%%%%%%%%%%%%%%%%%%%%%%%%%%%%%%%%%%%%%%%%%%%%%%%%%%%%%%%%%%%%%%%%%%%%%%%%%%%%%%%%%

\renewcommand\refname{Bibliografia}
\bibliographystyle{plain}
%\bibliography{bibliografia_wzor}

\end{document}
