\documentclass[12pt]{article}
\usepackage[T1]{fontenc}
\usepackage[T1]{polski}
\newcommand{\BibTeX}{{\sc Bib}\TeX}
\usepackage{graphicx}
\usepackage{siunitx}
\usepackage{hyperref}
\usepackage{amsmath} \usepackage{amssymb} \usepackage{amsfonts}
\usepackage[utf8]{inputenc}
\usepackage{minted}
\newcommand{\cljt}[1]{\mintinline{clojure}{#1}}
\usepackage{tikz-cd}
\setlength{\textheight}{21cm}

\title{{\bf Zadanie nr 1 - Generacja sygnału i szumu}\linebreak
	Cyfrowe Przetwarzanie Sygnałów}
\author{Jakub Mileczarek 236602 \and Antoni Jończyk, 236551}
\date{2023-03-13} % jak zdążymy xd

\begin{document}
\clearpage\maketitle
\thispagestyle{empty}
\newpage
\setcounter{page}{1}
\section{Cel zadania}
Celem zadania było zaimplementowanie programu pozwalającego na tworzenie,
wyświetlanie, zapisywanie i wczytywanie z plików sygnałów cyfrowych.
%Opis celu zadania (proszę nie przepisywać treści instrukcji!).\\
%Sprawozdanie należy wykonać na podstawie\\
%szablonu \LaTeX-owego \texttt{sprawozdanie-wzor.tex}.

%\section{Wstęp teoretyczny}

%Krótki opis wykorzystywanych metod~\cite{dowolna_etykieta_artykulu}. Proszę nie umieszczać ogólnie znanych z literatury
%wzorów oraz definicji. Należy podać jaka metoda została zastosowana, dlaczego oraz podać wykorzystaną literaturę (korzystając z odwołań do pozycji bibliografii~\cite{dowolna_etykieta_ksiazki}).\\
%Przygotowując bibliografię należy korzystać z podanego\\
%szablonu \BibTeX-owego \texttt{bibliografia-wzor.bib}.

%%%%%%%%%%%%%%%%%%%%%%%%%%%%%%%%%%%%%%%%%%%%%%%%%%%%%%%%%%%%%%%%%%%%%%%%%%%%%%%%%%%%%%%%%%%%%%%%%%%%%%%%%%%%%%%%%
% PODROZDZIA\xC5\x81 PT. ZALACZNIKI
%%%%%%%%%%%%%%%%%%%%%%%%%%%%%%%%%%%%%%%%%%%%%%%%%%%%%%%%%%%%%%%%%%%%%%%%%%%%%%%%%%%%%%%%%%%%%%%%%%%%%%%%%%%%%%%%%
\section{implementacja}
Program został zaimplementowany w \texttt{clojure}, funkcyjnym, lispopodobnym języku hostowanym na \texttt{JVM}. Do wykonania grafów użyliśmy biblioteki \texttt{incanter}{\cite{incanter}}, funkcję zwracającą wartość z rozkładem gaussowskim znaleźliśmy w bibliotece \text{cern.jet.random}, do wsparcia liczb złożonych wykorzystujemy pośrednio biblotekę \texttt{org.apache.commons.math3.complex}.
\subsection{reprezentacje}
Program korzysta wewnętrznie z 4 różnych reprezentacji sygnału:
\begin{itemize}
	\item		\cljt{:spec} \\
	      czytelny dla człowieka słownik zawierający specyfikację, większość elementów ma dość rozsądne wartości domyślne więc może być pominięta, n.p:\\
	      \cljt{{:fun :triangle :fill 0.4 :amplitude 0.4}} \\
	      \cljt{{:fun :sin :phase 0.3 :duration 3}} \\
	      dla wygody niektóre klucze posiadają skrócone wersje: \\
	      \cljt{{:f :sin :A 2 :p 3 :s 2 :e 5}}

	\item		\cljt{:fancy} \\
	      reprezentacja sygnału ciągłego jako słownik zawierający informacje o początku i końcu sygnału oraz funkcję zwracającą wartość sygnału w danym czasie
	\item		\cljt{:discrete} \\
	      zdyskretyzowana forma sygnału, zawiera informacje o częstotliwości próbkowania, offsecie, ilości próbek, i próbki
	\item		\cljt{:file} \\
	      nazwa pliku zawierającego sygnał
\end{itemize}
Funkcje do manipulacji wspierają jedynie \cljt{:discrete} i \cljt{:fancy} (nie wszystkie), inne reprezentacje są w miarę potrzeb konwertowane do którejś ze wspieranych form
\begin{figure}
	\[\begin{tikzcd}[%
				%row sep = 20mm, column sep = 20mm, % both optional, adapt as you please
				cells={},
				every arrow/.append style={shorten <= 1mm, shorten >= 1mm}] % both optional. Default connects the arrow directly to the circles
			:file \arrow[shift left]{r} &
			:discrete \arrow[shift left]{l} &
			:fancy \arrow{l} &
			:spec \arrow{l} &
		\end{tikzcd}\]
	\caption{różne reprezentacje i konwersje między nimi}
\end{figure}

Dodatkowym i być może nieoczekiwanym krokiem jest \cljt{:fancy}, jego celem jest maksymalne opóźnienie próbkowania, tak by można było np. dodać do sygnału z pliku sumę dwóch sygnałów, i spróbkowanie ich dopiero w momencie w którym znana jest jego częstotliwość próbkowania.

\subsection{częstotliwość}
Program stara się zachować częstotliwość sygnału, by przy tworzeniu histogramu uciąć wystającą część okresu. w przypadku wykonywania operacji na sygnale częstotliwość sygnału wynikowego jest najmniejszą wspólną wielokrotnością sygnałów stanowiących argumenty operacji.
\section{Instrukcja}
\subsection{wprowadzanie sygnałów}
Do uruchomiena interaktywnej sesji \texttt{REPL} służy komenda \texttt{lein repl}
Użytkownik wprowadza sygnały (poza impulsem jednostkowym) w formie \cljt{:spec} poprzez mapę zawierającą
chciane parametry
\begin{itemize}
	\item \cljt{:fun, :fun, :fn, :f} -- jedno z słów kluczowych oznaczających funkcje
	      \begin{itemize}
		      \item \cljt{:sin} sinusoida
		      \item \cljt{:sin-half} sinusoida wyprostowana jednopołówkowo
		      \item \cljt{:sin-double} sinusoida wyprostowana dwupołuwkowo
		      \item \cljt{:square} sygnał prostokątny (\cljt{:fill} -- stopień wypełnienia)
		      \item \cljt{:square-sym} sygnał prostokątny symetryczny
		      \item \cljt{:triangle} sygnał trójkątny (\cljt{:fill} -- część okresu przeznaczona na zbocze narastające)
		      \item \cljt{:jump} skok jednostkowy (\cljt{:fill} -- czas skoku)
		      \item \cljt{:noise} szum (-1 -- 1)
		      \item \cljt{:noise-gauss} szum gaussowski
		      \item \cljt{:noise-impulse} szum impulsowy (\cljt{:fill} -- prawdopodobieństwo)
	      \end{itemize}
	\item \cljt{:period, :per, :p} -- okres, wartość domyślna = 1
	\item \cljt{:amplitude :amp :a :A} -- amplituda, wartość domyślna = 1
	\item \cljt{:fill} -- znaczenie różne w zależności od funkcji, domyślnie 0.5

	\item \cljt{:phase, :ph} -- przesunięcie fazowe, domyślnie 0.0

	\item \cljt{:start, :s} -- czas początku, wyrażony w sekundach, domyślnie 0
	\item \cljt{:duration, :d, :l} -- czas trwania, wyrażony w sekundach, domyślnie 5
	\item \cljt{:end, :e} -- koniec, bez wartości domyślnej, jeżeli podany, jest brany pod uwagę zamiast \cljt{:duration}
\end{itemize}
Do wprowadzania impulsu jednostkowego służy funkcja \cljt{impulse} biorąca jako parametry \cljt{:A} --amplituda, domyślnie 1.0, \cljt{:A} --amplituda, domyślnie 1.0, sampling, domyślnie 1/1000,  \cljt{:ns} -- nr. próbki (liczony od 0) w której występuje impuls. przykład \cljt{(impulse :ns 5 :A 2.0)}
\subsection{operacje}
Do przesuwania sygnałów w czasie służy funkcja \cljt{(tshift x t)}, gdzie \cljt{x} to sygnał a \cljt{t} to czas. \\
Do operacji na sygnale służą funkcje \cljt{fop, dop}, pierwsza z nich działa na sygnałach typu \cljt{:fancy}, a druga na \cljt{:discrete}, automatycznie próbkując inne sygnały z odpowiednią częstotliwością. Sygnatura obu z nich jest taka sam \cljt{(fop funkcja sygnał1 sygnał2 sygnałn)}, \cljt{funkcja} musi zostanie wywołana z wartościami poszczególnych sygnałów w danym czasie. przykłady: \\
\cljt{(dop - "sygnał" {:fun :sin :d 1})} -- wczyta sygnał z pliku, dokona dyskretyzacji funkcji sinus na przedziale od 0 do 1 z częstotliwością taką samą jak ta wczytana z pliku, i dla każdej pary próbek wykona operację odejmowania, zwróci \cljt{:discrete} \\
\cljt{(fop * {:fun :sin} {:fun :sin :period 1/3})} -- zwróci sygnał w postaci \cljt{:fancy} którego wartości w poszczególnych chwilach czasowych zostaną obliczone później
\subsection{próbkowanie}
domyślna częstotliwość próbkowania jest zmienną dynamiczną, jej wartość można ustawić poprzez \\
\cljt{(binding [*sampling-frequency* 1/20)] kod)))}

\subsection{prezentacja}
Funkcje znajdujące się w przestrzeni nazw \cljt{graph}
\begin{itemize}
	\item \cljt{show} -- wyświetl sygnał jako punkty lub krzywą w zależności od typu sygnału
	\item \cljt{show-two} -- wyświetl dwa sygnały (konwertując drugi to tego samego typu co pierwszy)
	\item \cljt{histogram} -- wyświetla histogram sygnału dyskretnego
	\item \cljt{stat} -- zwraca mapę zawierającą parametry sygnału dyskretnego
	\item \cljt{other-stat} -- oblicza parametry sygnału,traktując pierwszy z
	      sygnałów jako oryginalny, a drugi jako odtworzony
\end{itemize}
Zachowanie tych funkcji można zmienić ustawiając odpowiednie zmienne dynamiczne
\begin{itemize}
	\item \cljt{*magnitude*} -- domyślnie \cljt{true}
	      (część rzeczywista i urojona), gdy zmieniona wyświetla moduł i kąt
	\item \cljt{*together*} -- decyduje czy oba wykresy mają być wyświetlone ,,na
	      sobe'' czy w oddzielnych oknach, domyślnie \cljt{tru}

	\item \cljt{*graph-samples*} maksymalna ilość wyświetlanych próbek
	\item \cljt{*hist-bins*} ilość binów w histogramie, domyślnie $20$
\end{itemize}

\subsection{zapisywanie i wczytywanie sygnałów do pliku}
Do zapisywania sygnału do pliku służy procedura \cljt{write} biorąca dwa argumenty, nazwę pliku i sygnał;
\cljt{(write "sygnał" {:fun :sin})}
wczytywanie odbywa się automatycznie;
\cljt{(show (fop + "sygnał" {:fun :square}))}.
\subsection{konwolucja i filtrowanie}
w przestrzeni nazw \cljt{signal} znajdują się dwie funkcje \cljt{convolute} i
\cljt{correlate}, przyjmują one dwa sygnały i ich zachowanie jest zgodne z nazwą
w przestrzeni nazw \cljt{filter} znajdują się funkcje mające na celu ułatwienie
tworzenia filtrów, najużyteczniejszymi z nich są:
\begin{itemize}
	\item \cljt{make-filter}, przyjmuje ona mapę z czterema argumentami, \cljt{M}, \cljt{K}, \cljt{pass}
	      i \cljt{window}, gdzie pass to jedna z funkcji \cljt{middle}, \cljt{upper}
	      \cljt{lower}, a \cljt{window} to jedno z okien \cljt{square},
	      \cljt{hamming}, \cljt{blackman}, \cljt{hanning}, zwraca filtr
	\item \cljt{make-lower-pass-filter}, przyjmuje mapę zawierającą inne parametry
	      (M i okno) oraz częstotliwość odcięcia, zwraca filtr dolnopasmowy
	\item \cljt{make-upper-pass-filter}, przyjmuje mapę zawierającą inne parametry
	      (M i okno) oraz częstotliwość odcięcia, zwraca filtr górnopasmowy
	\item \cljt{make-pass-filter}, przyjmuje mapę zawierającą inne parametry
	      (M i okno) oraz dolną i górną częstotliwości odcięcia, zwraca filtr pasmowy
	\item \cljt{filter-stat}, przyjmuje krok, i filtr, filtruje
	      sygnały o różnych częstotliwościach, i zwraca sygnał dyskretny
	      reprezentujący stopień w jakim przepuszcza je filtr, im mniejszy krok
	      tym więcej próbek
\end{itemize}
\subsection{transformaty}
w przestrzeni nazw \cljt{trans} znajdują się następujące funkcje:
\begin{itemize}
	\item \cljt{fourier-slow} -- transformata fouriera
	\item \cljt{fourier-fast} -- algorytm szybki, w miejscu, z partycjonowaniem w czasie
	\item \cljt{fourier-frequency-fast} -- jak wyżej, tylko partycjonowanie jest w przestrzeni
	\item \cljt{kosinus-slow} -- transformata kosinusowa
	\item \cljt{kosinus-fast} -- jak wyżej, z wykorzystaniem fft
	\item \cljt{hadamard-slow} -- mnoży sygnał przez macież Hadamarda
	\item \cljt{hadamard-fast} -- szybka implementacja powyższego algorytmu
\end{itemize}
wszystkie mają też wariant kończący się na \cljt{-rev} odwracający
transformację.
w przypadku nierównej 2^n ilości próbek, ostatnie próbki są ucinane.
funkcja \cljt{wavelet} zwraca wektor dwóch sygnałów
\section{wyniki}
porównanie szybkości
\begin{table}[H]
	\centering
	\caption{porównanie szybkości implementacji funkcji  fourier  (czas w ms)}
	\begin{tabular}{| S|S|S| } \hline
		{n}  & {wolna}     & {szybka}  \\ \hline
		2    & 3.154699    & 0.465392  \\ \hline
		4    & 3.778578    & 0.34676   \\ \hline
		8    & 7.189706    & 0.449263  \\ \hline
		16   & 11.481976   & 0.74542   \\ \hline
		32   & 26.061494   & 1.491214  \\ \hline
		64   & 70.893625   & 2.599292  \\ \hline
		128  & 203.092074  & 5.249943  \\ \hline
		256  & 689.74614   & 10.273186 \\ \hline
		512  & 2481.301696 & 20.962731 \\ \hline
		1024 & 9507.214255 & 46.574992 \\ \hline
	\end{tabular}
\end{table}

\begin{table}[H]
	\centering
	\caption{porównanie szybkości implementacji funkcji  kosinus  (czas w ms)}
	\begin{tabular}{| S|S|S| } \hline
		{n}  & {wolna}      & {szybka}  \\ \hline
		2    & 2.139726     & 0.574628  \\ \hline
		4    & 3.128492     & 0.736451  \\ \hline
		8    & 7.346126     & 0.955503  \\ \hline
		16   & 12.724274    & 1.573441  \\ \hline
		32   & 30.084534    & 2.647535  \\ \hline
		64   & 93.293692    & 5.222674  \\ \hline
		128  & 249.427242   & 15.387668 \\ \hline
		256  & 857.811608   & 19.159811 \\ \hline
		512  & 2957.385779  & 34.956977 \\ \hline
		1024 & 11104.883967 & 76.9784   \\ \hline
	\end{tabular}
\end{table}

\begin{table}[H]
\centering
\caption{porównanie szybkości implementacji funkcji  hadamard  (czas w ms}
\begin{tabular}{| S|S|S| } \hline
{n} & {wolna} & {szybka} \\ \hline
2 & 0.508474 & 0.070081 \\ \hline
4 & 0.736516 & 0.071407 \\ \hline
8 & 2.436082 & 0.086338 \\ \hline
16 & 12.033876 & 0.174339 \\ \hline
32 & 39.674842 & 0.290743 \\ \hline
64 & 160.804605 & 0.408094 \\ \hline
128 & 628.61702 & 0.792157 \\ \hline
256 & 2544.785067 & 1.780247 \\ \hline
512 & 10146.765121 & 3.346331 \\ \hline
1024 & 40880.995511 & 7.521443 \\ \hline
\end{tabular}
\end{table}



\newpage
%%%%%%%%%%%%%%%%%%%%%%%%%%%%%%%%%%%%%%%%%%%%%%%%%%%%%%%%%%%%%%%%%%%%%%%%%%%%%%%%%%%%%%%%%%%%%%%%%%%%%%%%%%%%%%%%%
% BIBLIOGRAFIA
%%%%%%%%%%%%%%%%%%%%%%%%%%%%%%%%%%%%%%%%%%%%%%%%%%%%%%%%%%%%%%%%%%%%%%%%%%%%%%%%%%%%%%%%%%%%%%%%%%%%%%%%%%%%%%%%%
\cite{instrukcja}
\renewcommand\refname{Bibliografia}
\bibliographystyle{plain}
\bibliography{ich}

\end{document}
